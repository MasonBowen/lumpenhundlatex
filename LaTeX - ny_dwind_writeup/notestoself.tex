Results to add:
\begin{itemize}
\item are there any adaptations in VDER necessary for distr. wind?
\item what vder values drive wind?
\item where is the potential concentrated and why?
\end{itemize}

Figures to add:
\begin{itemize}
\item wind resource map as seen in ES of \cite{mccabe_assessment_2018}.
\item fix maps to include municipal utility territory and merge con edison territory
\end{itemize}

Introduction notes:
\begin{itemize}
\item talk about Ben/Kevin's paper \citet*{mccabe_assessment_2018}.
\item more background into value of solar \citet*{denholm_methods_2014}
\item more background into NY
\item differentiating between different sizes of distributed wind
\item this paper differs from previous dGen analyses as it does not consider future developments/progressions, but only a snapshot of a single year analysing all sites, not just economic ones
\end{itemize}

Bringing together techno-economic potential, granular compensation mechanisms and distributed wind:
\begin{enumerate}
\item bass diffussion curve basics (starts in small pockets, then spreads to general public).
\item more granular analysis can help identify in few pockets where distr. wind profitable
\item solar provides a marginal benefit as it exhibits coincident generation
\item granular compensation mechanisms based on value to system can reward select pockets of wind and
\item are more likely to trend towards rewarding non-solar resources as power system saturates (assumes no measures taken to shift solar energy or demand)
\end{enumerate}

\vspace{25mm}
